\documentclass[12pt,a4paper,openbib]{article}
\usepackage[utf8]{inputenc}
\usepackage[spanish]{babel}
\usepackage{amsmath}
\usepackage{amsfonts}
\usepackage{amssymb}
\author{Inoro}
\title{Falacias y argumentos}
\begin{document}

	\section{FALACIAS Y ARGUMENTOS}
	
	Listado de tipos de falacias y argumentos explicados para tenerlos en cuenta en futuras argumentaciones sobre temas.
	
	\subsection{Falacias}
	
	Técnicas ilícitas que se deben evitar cuando se discuten argumentos.
	
	\begin{itemize}
		\item \textbf{Caricaturización y lenguaje emotivo:} referirte con tono burlesco a las ideas de tu oponente.
		\item \textbf{La falacia del monstruo:} Relacionar la opinión del contrario con algo muy malo, conocido por todos, para hacer parecer sus ideas irrazonables.
		\item \textbf{Generalización excesiva:} Estirar la afirmación del oponente hasta hacerle decir algo que el no ha dicho.
		\item \textbf{Olvido de alternativas:} pensar que todas las opciones se reducen solo a dos, o a muy pocas.
			\begin{itemize}
				\item \textbf{Variante del falso dilema:} reducir todas las opciones a dos, que se plantean de una manera muy dramática.
			\end{itemize}
		\item \textbf{Falacia \textit{ad hominem}:} atacar a la persona, en lugar de atacar a sus argumentos.
	\end{itemize}
	
	\subsection{Argumentos}
	
	\begin{itemize}
		\item a.
	\end{itemize}

\end{document}