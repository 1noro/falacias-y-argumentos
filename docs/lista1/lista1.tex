\documentclass[12pt,a4paper,openbib]{article}
\usepackage[utf8]{inputenc}
\usepackage[spanish]{babel}
\usepackage{amsmath}
\usepackage{amsfonts}
\usepackage{amssymb}
\author{Inoro}
\title{Falacias y argumentos}

%---------------------------------------------------------------------------------

\setlength\parindent{0pt} %PARA QUE NO SE AUTOINDENTE LA PRIMERA LINEA DE CADA PARRAFO
\setlength{\parskip}{0.2em} %SEPARACIÓN ENTRE PÁRRAFOS

%---------------------------------------------------------------------------------

\begin{document}

	\begin{center}
		\textbf{\Large FALACIAS Y ARGUMENTOS}\bigskip
	\end{center}
	
	Listado de tipos de falacias y argumentos explicados para tenerlos en cuenta en futuras argumentaciones sobre temas variados.
	
	\section{Falacias}
	
	Técnicas ilícitas que se deben evitar cuando se discuten argumentos.
	
	\paragraph{Caricaturización y lenguaje emotivo:} Referirte con tono burlesco a las ideas de tu oponente.
	
	\paragraph{La falacia del monstruo:} Relacionar la opinión del contrario con algo muy malo, conocido por todos, para hacer parecer sus ideas irrazonables.
	
	\paragraph{Generalización excesiva:} Estirar la afirmación del oponente hasta hacerle decir algo que el no ha dicho.
	
	\paragraph{Olvido de alternativas:} Pensar que todas las opciones se reducen solo a dos, o a muy pocas.
	
	\subparagraph{Variante del falso dilema:} Reducir todas las opciones a dos, que se plantean de una manera muy dramática.
	
	\paragraph{Falacia \emph{ad hominem}:} Atacar a la persona, en lugar de atacar a sus argumentos.
	
	\paragraph{Falacia de alegato especial:}
	
	\section{Argumentos}
	
	\begin{itemize}
		\item a.
	\end{itemize}
	
	\section{Sesgos cognitivos}
	
	Efecto psicológico que produce una desviación en el procesamiento mental, lo que lleva a una distorsión, juicio inexacto, interpretación ilógica, o lo que se llama en términos generales irracionalidad, que se da sobre la base de la interpretación de la información disponible, aunque los datos no sean lógicos o no estén relacionados entre sí.
	
	\paragraph{Efecto Dunning-Kruger:} Los individuos con escasa habilidad o conocimientos sufren de un sentimiento de superioridad ilusorio, considerándose más inteligentes que otras personas más preparadas, midiendo incorrectamente su habilidad por encima de lo real.
	
	Lo opuesto al \emph{efecto Dunning-Kruger} es la \emph{falacia de alegato especial}.

\end{document}